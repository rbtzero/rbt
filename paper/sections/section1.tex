\section{Introduction \& Context}
\label{sec:intro}
% TODO: write historical motivation and roadmap.

Recursive Becoming sits at the crossroads of information theory and high-energy physics.  Over the past century physicists have progressively revealed a unification arc: Maxwell stitched together electricity and magnetism; Einstein welded space with time and then gravitation; Weinberg, Salam, and Glashow fused electroweak interactions; and quantum chromodynamics completed the Standard Model.  Each unification step compressed previously distinct concepts into deeper symmetry principles, shrinking the number of free parameters required to encode empirical data.

The next logical step is radical: eliminate \\emph{all} external parameters.  Could the fundamental constants themselves emerge from a self-referential dynamical law rather than being imposed as inputs?  Inspired by Wheeler's "it from bit" dictum, RBT answers in the affirmative.  The theory begins with a single irreversible observation---one bit of information whose existence cannot be undone.  That bit catalyses a deterministic counting procedure that records its own history, generating an ever-growing ledger of states.  Algebraic patterns on this ledger manifest as conserved quantities, metric structure, and eventually the familiar quantum fields of low-energy physics.

This paper has two goals.  First, we provide a minimal axiomatic presentation of RBT that exposes \texttt{bit->algebra->geometry->physics} as a logically necessary cascade.  Second, we present the first numerical experiment exploring ten thousand recursive steps, validating the analytic claims and uncovering unexpected phenomena such as spontaneous chemical--like autocatalysis.  Figure~\ref{fig:intro-timeline} situates RBT within the historical march of unification.

\begin{figure}[h]
  \centering
  \includegraphics[width=0.8\linewidth]{figs/intro_timeline.pdf}
  \caption{Milestones in physical unification leading to Recursive Becoming.}
  \label{fig:intro-timeline}
\end{figure}
\clearpage 