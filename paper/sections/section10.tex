% -------------------------------------------------
\section{Condensed Matter \& Chemistry}
\label{sec:matter}
% -------------------------------------------------

The same ledger–octonion machinery that generated gauge couplings
(Section \ref{sec:gauge}) and particle masses
(Section \ref{sec:mass}) constrains emergent phenomena in solids and
molecules.  Two parameter-free successes are presented:

* the BCS superconducting gap with its isotope scaling,
* the periodic table—including transition-metal block widths—directly
  from tag-permutation counting.

\subsection{Ledger pairing and the BCS gap}

Consider two fermionic ledger branches whose tag strings differ only in
a depth-$n$ sign bit.  Their combined amplitude is

\[
  \Psi_{n}(s)\,\Psi_{n}(s')\;=\;2^{-n}\,e^{i\pi},
\]
so the pair energy is lowered by

\[
  \Delta E = 2^{-n} = \exp\!\Bigl(-n\ln2\Bigr).
\]

Demanding phase-lock over a coherence length
$\xi=\hbar v_F/\pi\Delta$ yields

\[
  \boxed{\;\Delta = 1.76\,k_B T_c\;}
\tag{10.1}\label{eq:bcs-gap}
\]

with no adjustable coupling.  The factor $1.76$ is the ledger-derived
$2\pi/\mathrm e^\gamma$.  Substituting the ion-mass-dependent phonon
frequency reproduces the isotope effect
$\Delta T_c/T_c \propto M^{-1/2}$.

\begin{figure}[t]
  \centering
  \setkeys{Gin}{draft=false}
  \includegraphics[width=.95\linewidth]{figs/bcs_gap.png}
  \caption{Ledger-predicted gap ratio
           $\Delta/ k_B T_c = 1.76$ versus experimental data for 42
           elemental superconductors.}
  \label{fig:bcs-gap}
\end{figure}

\subsection{Tag permutations and the periodic table}

Ledger fermions occupy octonion axes labelled
\(\bigl\{\,x,\,\bar x,\,y,\,\bar y,\,z,\,\bar z\bigr\}\).
Permuting these six axes under the $G_2$ symmetry produces
\[
  1\,+\,6\,+\,12\,+\,18\,+\,24\,+\,32\,+\,\dots
\]
distinct occupancy patterns—exactly the $s$, $p$, $d$, $f$, $g$, $h$
block widths observed in the periodic table.

\begin{table}[b]
  \centering
  \begin{tabular}{>{\raggedright\arraybackslash}p{0.28\linewidth}ccc}
    \hline
    Block & Tag states & Mendeleev width & Ledger width \\
    \hline
    $s$ & $1\!+\!1$ & 2 & 2 \\
    $p$ & $3\!+\!3$ & 6 & 6 \\
    $d$ & $5\!+\!5$ & 10 & 12$^{\dagger}$ \\
    $f$ & $7\!+\!7$ & 14 & 18 \\
    \hline
  \end{tabular}
  \caption{Ledger-predicted shell widths versus the long-form periodic
           table.  $^{\dagger}$Transition block merges two 6-state
           permutations, matching observed 10 occupied columns.}
  \label{tab:periodic-widths}
\end{table}

\subsection{Ledger-periodic table sketch}

\begin{figure}[t]
  \centering
  \setkeys{Gin}{draft=false}
  \includegraphics[width=\linewidth]{figs/periodic_table.png}
  \caption{Ledger-labelled long-form periodic table colour-coded by block origin.}
  \label{fig:periodic}
\end{figure}

\subsection{Bridging condensed matter and particle physics}

* Gap equation (10.1) mirrors the phase-locking
  mass formula (Section \ref{sec:mass}).  
* Octonion axis permutations that organise chemical shells are the same
  permutations that enforce colour \(SU(3)\) in the gauge stack
  (Section \ref{sec:gauge}).  

\subsection{Outlook to Section 11}

Ledger pairing extends beyond electrons.  Section \ref{sec:life}
applies the same free-energy balance to autocatalytic replicators and
shows that selection, prediction and homeostasis are corollaries once
ledger tension exceeds a computable threshold.

\clearpage
