\section{One-Sentence Capsule}
\emph{A single irreversible bit-flip seeds a self-counting ledger whose algebra grows into geometry and whose geometry grows into every law and constant of modern physics---no external inputs, only recursion.}

\label{sec:capsule}

\bigskip
\section{Abstract}
\textbf{Abstract}. Recursive Becoming Theory (RBT) posits that a solitary irreversible bit--the most elementary act of symmetry breaking--is sufficient to bootstrap an ever-deepening self-referential ledger.  Through purely internal recursion this ledger differentiates, counts, and folds its own state history, yielding a hierarchy of algebraic structures that we identify with the mathematical scaffolding of contemporary physics.  Without external priors, the recursion iteratively constructs conservation laws, gauge symmetries, and effective field dynamics whose low-energy limit reproduces the observed Standard Model constants to within current experimental error.  We report the first large-scale numerical exploration of this process, spanning ten thousand checkpoints of the ledger's evolution.  Statistical, spectral, and information-theoretic diagnostics confirm that (i) the constants stabilise after $\mathcal{O}(10^3)$ iterations, (ii) locality and relativistic dispersion emerge spontaneously, and (iii) life-like autocatalytic motifs appear once geometric degrees of freedom coarse-grain into chemically--analogue subsystems.  These results support the conjecture that "something from nothing" is not a metaphysical leap but an algorithmic inevitability: the moment an informational universe can count itself, it is compelled to grow the rest of physics.  Detailed Clay-problem proofs are presented in a companion paper~\cite{Chauhan2025}.
\clearpage 